\chyph                      % české vzory dělení slov
\fontfam[Schola]            % fontová rodina TeXGyre Schola
\margins/1 a4 (3,3,3,3)cm   % okraje 3cm ze všech stran pro stranu A4
\typosize[11/13]            % velikost písma 11pt na 13pt řádkování
\activettchar"              % znak " bude ohraničovat řádkový verbatim
\hyperlinks\Blue\Green      % interní odkazy modré, webové odkazy zelené
\outlines0                  % klikací obsah po straně PDF prohlížeče

\def\lorem{\lipsumtext[1]}  % lorem ipsum generátor

\def\center{\noindent\hfil}  % zkratka pro centrování



\tit Ukázka -- \OpTeX/

\nonum\notoc\sec Obsah

\maketoc



\sec Z ákladní struktura

Dokument se skládá z odstavců (ve zdrojovém textu jsou odděleny prázdnými řádky)
a dále z kapitol, sekcí a podsekcí. \lorem

\lorem

\secc Výčty

V textu můžeme používat výčty:

\begitems
* první
* druhý
\enditems

\lorem

Někdy mohou být výčty vnořeny:

\begitems
* \lorem
* Druhý.
  \begitems \style a
  * Podproblém,
  * ještě jeden podproblém,
  * poslední podproblém.
  \enditems
* Třetí.
* Poslední.
\enditems
 
\lorem

\secc Vyznačování

Text můžeme zdůraznit {\it kurzívou}, nebo {\bf tučně}. \lorem

\secc[poznamky]  Poznámky

Poznámka pod čarou\fnote{Vypadá takto.} je automaticky číslovaná.
Může jich být více\fnote{Druhá poznámka.}.
\lorem\ \lorem

\sec[dalsi] Další sekce

\lorem

\secc Verbatim text

Vložíme {\it verbatim text}, tedy text \uv{tak jak je} bez interpretace
\TeX{}em:

\begtt
int main (void) {
  printf ("helo world!\n");
  return 0;
}
\endtt
Kromě toho je možné psát "#kusy {kódů}" přímo v textu odstavce.

\lorem


Obrázky se do odborného textu obvykle vkládají včetně popisku.

\midinsert \label[cmelak]
  \center \picw=7cm \inspic cmelak.jpg
  \medskip
  \caption/f Bombus sylvarum.
\endinsert

\lorem

\secc Odkazy

Na automaticky číslované objekty je možné odkazovat. Ve zdrojovém dokumentu
je třeba použít {\it lejblík} v místě cíle a stejný lejblík v místě odkazu.
Odkazy vypadají například takto: viz sekci~\ref[dalsi] na
straně~\pgref[dalsi], dále v podsekci~\ref[poznamky] je zmínka o poznámkách
pod čarou a rovnice~\ref[rce] na straně~\pgref[rce] uvádí zajímavou
matematickou myšlenku. Každý jistě poznal, že na obrázku~\ref[cmelak] se
nachází čmelák.

Vnější odkazy směřují obvykle na webové stránky, 
napÅ™.~\url{http://petr.olsak.net/opmac.html}.

\secc Tabulky

Na tabulky se používá "\hbox" / "\vbox" aritmetika \TeX{}u a 
příkaz "\halign". To ale není nic vhodného pro začátečníka. Proto makra
(OPmac, \LaTeX, Con\TeX{}t) umožňují autorům textu psát tabulky jednodušeji. 
Příklad tabulky: 

\bigskip
\center\table{||l|cc||}{\crl \tskip.5ex
           & \multispan2 \hfil počet \qquad\hfil\vrule\kern\vvkern\vrule\strut\cr
   makro             & řádků kódu & stran dokumentace \crli \tskip.5ex
   plain             &    1241 & 100 \cr
   OPmac             &    1600 &  19 \crli \tskip.5ex
   \LaTeX{} (jádro)  &    8000 & 240 \cr
   hyperref          &   24773 & 423 \cr
   color + graphics  &    2469 &  81 \crli \tskip.5ex
   Ti{\it k}Z        & 1\,634\,882 & 726 \crli \tskip.5ex
   Con\TeX{}t        & 1\,155\,673 & 547 \crl
 } 
\nobreak\medskip
\caption/t Makra a jejich rámcový rozsah.
\bigskip

V tabulce ve zdrojovém textu je jednak záhlaví s pravidlem, jak mají být
formátována jednotlivá pole tabulky, a dále data oddělená od sebe smluvenými
oddělovači. Na úrovni primitivního příkazu "\halign" jsou těmito oddělovači
znaky "&" a "\cr". Jednotlivá makra mohou autorům nabízet mírné modifikace
tohoto způsobu vyznačování.

\secc Matematika

V řádku se píše matematika mezi dolary: $a^2+b^2=c^2$. Nebo je možno sestavit
samostatnou rovnici obklopenou dvojicí dolarů:

\label[rce]
$$
  \int x^\alpha {\rm d}x = \cases {
     x^{\alpha+1} \over \alpha+1 & pro $\alpha\not=-1$ \cr \noalign{\medskip}
     \ln x                       & pro $\alpha=-1 $
  }
  \eqmark
$$ 
Vzorce se zapisují zhruba tak, jak bychom je anglicky přečetli.

\lorem

$$
  {\bbchar A} = \pmatrix{ a & b & c \cr d & e & f \cr g & h & i}.
$$
Toť na úvod vše.

\bye
