\fontfam[Termes]

\typosize[11/13]
\margins/1 a4 (1,1,1,1)in
\enlang
\nopagenumbers

\hyperlinks \Blue \Blue


\def\subl{Sublime Text}


\tit Sublime Text \OpTeX/ package

\hfil Daniel Pilař \hfil\par
\medskip
\hfil \the\year \hfil


\nonum\sec Installation

To use this package, you need to have \subl\ installed, preferably the newest version
({\it version 4, older versions were not tested and might cause problems}). If you don't have any \subl\ versions installed yet,
you can follow instructions on \subl\ \ulink[https://www.sublimetext.com]{official site}.  \par

Although there are several ways to publish packages using various packaging tools, the easiest way to use this package is to
move the package folder to {\tt Packages} directory. The path could be different on each platform, so to find the {\tt Packages} directory,
hit {\tt Preferences} in the top panel in \subl\ and select {\tt Settings}. \subl\ opens two settings files which are stored in {\tt Packages}
directory so you can see the path in the top panel.

\nonum\sec Features

\nonum\secc Syntax highlighting

The package provides a very simple syntax recognition using {\tt .sublime-syntax} files to highlight \TeX\ primitive commands, \OpTeX\ and
other macros. It also supports \OpTeX-math control sequences inside inline and block math sections. Comments and parameters inside square
brackets are also distiguishble from other text. If the OpTeX syntax is not selected automatically, go to {\tt View->Syntax} in the top panel
and choose OpTeX.

\nonum\secc Completion

Basic completions \TeX/ primitives and \OpTeX/ macros are also supported by plugin, which uses the \subl\ API written in Python. It also suggests
user-defined macros.

\nonum\secc Snippets

Snippets are generally used as smarter completions. This package supports the most simple snippets for inserting dolars and double dolars
surrounding the inline and block math sections and block verbatim macro pairs. The snippets can be triggered from the completion suggestion list
, from the command palette or when you type enough text to match particular snippet pattern and press {\tt Tab}. Snippets are defined in
{\tt .sublime-snippet} files.

\nonum\secc Simple build system

Sublime Text can use {\tt .sublime-build} files to provide simple build systems for your projects. This particular build system runs 
the optex command with the name of the file that is being edited. To trigger the build process, press {\tt Ctrl+B} or choose {\tt Tools->Build} in
the top panel.

\nonum\sec Used resources

\begitems
*\url{https://www.sublimetext.com/docs/api_reference.html}
*\url{https://www.sublimetext.com/docs/index.html}
*\url{https://raw.githubusercontent.com/kkos/oniguruma/5.9.6/doc/RE}
*\url{https://www.youtube.com/watch?v=Iqw32MO_A7E\&list=PLGfKZJVuHW91zln4ADyZA3sxGEmq32Wse}
*\url{petr.olsak.net/ftp/olsak/optex/optex-doc.pdf}
*\url{http://petr.olsak.net/ftp/olsak/optex/optex-math.pdf}
*\url{https://www.tug.org/utilities/plain/cseq.html}
\enditems

\bye